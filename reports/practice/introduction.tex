\subsection{Цели и задачи практики}
Цель данной учебной практики заключалась в подготовке исследования в рамках Выпускной квалификационной работы на тему "Атаки на мультиязычные модели". \\
Для успешного прохождения практики были поставлены следующие задачи:
\begin{enumerate}
    \item Обучение различных мультиязычных языковых моделей на датасете ATIS - Seven languages~\cite{Xu2020EndtoEndSA}.
    \item Генерация дополнительных тестовых выборок с помощью различных методов адверсариальных атак.
    \item Сравнение полученных результатов на всех выборках и анализ адверсариальных атак.
\end{enumerate}

\subsection{Постановка задачи}
Основная задача практики заключается в анализе различных адверсариальных атак на мультиязычные языковые модели.
Модели должны быть обучены на датасете ATIS - Seven languages ~\cite{Xu2020EndtoEndSA} для задачи одновременного выделения слотов и классификации интентов пользователя.
\newline
Каждая из рассматриваемых адверсариальных атак состоит в различных пертурбациях тестовой выборки, а именно смешении языков внутри одного предложения.
Результатом практики будет сравнение полученных результатов для каждой из атак для каждой из моделей.
\newline
Если у нас есть целевая модель $\mathcal{M}$, пример из тестовой выборки $x$ с метками $y$, то наша цель найти такую пертурбацию $x$, которая максимизирует ошибку модели $\mathcal{M}$.
\[
    x' = \argmax\limits_{x_c \in X} \mathcal{L}\left( y, \mathcal{M}(x_c) \right),
\]
где $x_c \in X$ это адверсариальная пертурбация $x$.

\subsection{Актуальность темы}
В последнее время создаётся всё больше мультиязычных моделей и появляется всё больше исследований на тему межъязыкового обобщения.
Новые методы и модели показывают впечатляющие результаты в переносе знаний и дообучении.
Однако перенос с одного языка на другой недостаточен для таких моделей для полноценного понимания мультиязычных людей в мультиязычных сообществах по всему миру.
Во многих из этих сообществ смешение языков внутри одного предложения или фразы является повседневной практикой.
Это называется код-свитчинг, феномен специфичный для мультиязычных сред, возникающий как в обычных разговор, так и в переписках и постах в интернете.
Таким образом, для систем обработки естественного языка важно уметь работать и показывать хорошее качество на таких входных данных.
Существуют вручную собранные и размеченные датасеты с код-свитчингом, которые позволяют оценить реальное качество моделей и дообучить их.
Но сбор и разметка таких датасетов очень дорогие, так же возможное количество смешений различных языков является большой проблемой.
\newline
Мы постулируем, что качество модели на искусственно сгенерированных с использованием адверсариальных атак тестовых данных может служить нижней оценкой на качество модели на реальных данных с код-свитчингом.
Эти сгенерированные данные будут служить "самым плохим случаем", что позволит думать, что в случае реальных данных качество модели будет выше.
